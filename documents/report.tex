\documentclass[12pt]{article}
	\usepackage[utf8]{inputenc}
	
	\usepackage{url}
	\usepackage{lipsum}
	\usepackage{graphicx}
	\usepackage{color}
	\usepackage{amsmath}
	\usepackage{amssymb}
	\usepackage{accents}
	\usepackage{tikz}
	\usepackage[backend=bibtex,style=numeric]{biblatex}
	\usepackage{algpseudocode}
	\usepackage{hyperref}
	\addbibresource{bibliografia.bib}
	\DeclareMathOperator*{\argmin}{arg\,min}
	\title{
		\vspace{-30mm}\begin{figure}[h]
			\centering
			\includegraphics[width=2in]{logo_polimi.png}
			% \label{escudo}
		\end{figure}Anderson Acceleration for Fixed-Point Iterations}
		\author{Martino Ischia\\ \footnotesize{Supervisor: Prof. Formaggia}}
		\date{January 2021}
		
		
		\begin{document}
		\maketitle
		\abstract
			The goal of this project is twofold.\\
			Firstly, develop a C++ interface for accelerating a converging sequence of vectors.
			It should be at the same time open for extensions, in case the user wants to implement its own algorithm,
			and efficient in the solution of large systems.\\
			Secondly, making use of said interface to implement Anderson acceleration strategy and apply it on ...\\
			
			The reader not interested in the C++ implementation, exposed in section \ref{sec:C++},
			can safely skip it.
		\endabstract
		\tableofcontents
		
		\section{Introduction}
			A fixed-point problem consists in finding the point (a vector in $\mathbb{R}^n$) that satisfies the
			following:
			$$ x = g(x)$$
			where g is a function from $\mathbb{R}^n$ to $\mathbb{R}^n$.\\
			It is clearly equivalent to the problem of finding the roots of a generic function,
			that is often tackled by employing Newton-Raphson iterative algorithm.
			In many practical problems, though, the cost of computing the Jacobian of a function
			is not practical, not to mention that there might be problems in starting the
			iterations from a proper guess.\\
			Considering the fixed-point form and solving through fixed-point iterations has also some limitations, the main one being
			a slow convergence: in fact most of the times the convergence is linear.\\
			Several strategies have been described in the literature to improve the speed
			of convergence of a vector sequence: we could refer to them as acceleration methods.
			This project focuses on one of them,
			proposed by \citeauthor{Anderson} \cite{Anderson} in \citeyear{Anderson}.
			
			\subsection{Anderson Algorithm}
				There are many equivalent ways to formulate Anderson acceleration method.
				I report the one in \citeauthor{Walker} \cite{Walker}.\\
				Several adjustment can be made to address specific problems, but .....
				
				
				\begin{algorithmic}
					\State Given $x_0$ and $m \geq 1$
					\State Set $x_1 = g(x_0)$
					\For {$k = 1, 2, ...$ until convergence}
					\State Set $m_k = min\{m, k\}$
					\State Set $F_k = (f_{k-m_k}, ... , f_k)$, where $f_i = g(x_i)-x_{i}$
					\State Determine $\alpha^{(k)} = (\alpha^{(k)}_0 , ..., \alpha^{(k)}_{m_k} )^T$, subject to 
					$\sum^{m_k}_{i=0} {\alpha_i = 1}$, that solves
					$$\min_{\alpha=(\alpha_0,...,\alpha_{m_k} )^T} \|F_k \alpha\|_2$$
					\State Set $x_{k+1} = (1-\beta) \sum^{m_k}_{i=0} {\alpha_i^{(k)} x_{k-m_{k}+i}}	+\beta \sum^{m_k}_{i=0} {\alpha_i^{(k)} g(x_{k-m_{k}+i})}$   
					\EndFor
				\end{algorithmic}
				
			The new value is obtained through a linear combination of the previous iterates and their evaluations.
			The coefficients are found through a minimization problem, in this case written as a constrained
			problem, but an unconstrained form is usually chosen for the implementation \cite{Fang}\cite{Walker}.\\
			Notice that the algorithm depends on two parameters: $\beta$, a relaxation parameter that has also
			a special interpretation when the algorithm is seen as a multisecant method \cite{Fang}, and $m$,
			a memory parameter, that represents the number of previous iterates considered in the calculations.
			
				
				
		\section{C++ interface for a generic accelerator}
		\label{sec:C++}
			aaaa
		\section{FEM application}
		\label{sec:FEM}
			aaa

						
			\printbibliography[heading=bibintoc,
			title={Bibliography}]
						
					
	\end{document}
								